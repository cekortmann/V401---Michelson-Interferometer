\section{Diskussion}
\label{sec:Diskussion}



Bei der Messung des Brechungsindex  von Luft ist die sehr kleine Fehlerabweichung gegenüber des eigentlichen Werts bemerkenswert. Wenn dieser nun 
mit dem Literaturwert $n_{\symup{Luft}} = 1.0003$ \cite{brechung} verglichen wird, ergibt sich eine Abweichung von $0.01\,\%$. Das zeigt, dass das Michelson-Interfermeter 
eine gute Messapparatur ist, um den Brechungsindex zu bestimmen. Da in diesem Versuchsteil der Spiegel nicht verschoben wurde, wird eine große Fehlerquelle 
ausgeschlossen. In \autoref{eqn:brechung} dominiert jedoch der Druckterm, welcher sehr unanfällig für Fehler ist.