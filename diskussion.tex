\section{Diskussion}
\label{sec:Diskussion}

Die ermittelte Wellenlänge hat den Wert $(736.0 \pm 1.5)10^{-9}\, \unit{\meter}$ und damit Abweichung von $(15.91 \pm 0.24)\%$ zum Theoriewert, welcher bei $635.0\cdot 10^{-9}\, \unit{\meter}$ liegt.
Die Abweichung könnte durch Verwacklungen der Apparatur oder durch das kurzzeitige Aussetzen des Messgerätes begründet werden. 
Durch einen vorsichtigeren Umgang mit der Messapparatur oder die Verwendung eines intakten Messgerätes könnte die Abweichung reduziert werden.

Bei der Messung des Brechungsindex  von Luft ist die sehr kleine Fehlerabweichung gegenüber des eigentlichen Werts bemerkenswert. Wenn dieser nun 
mit dem Literaturwert $n_{\symup{Luft}} = 1.0003$ \cite{brechung} verglichen wird, ergibt sich eine Abweichung von $0.01\,\%$. Das zeigt, dass das Michelson-Interfermeter 
eine gute Messapparatur ist, um den Brechungsindex zu bestimmen. Da in diesem Versuchsteil der Spiegel nicht verschoben wurde, wird eine große Fehlerquelle 
ausgeschlossen. In \autoref{eqn:brechung} dominiert jedoch der Druckterm, welcher sehr unanfällig für Fehler ist.