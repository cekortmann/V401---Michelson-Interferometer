\section{Auswertung}
\label{sec:Auswertung}

\subsection{Fehlerrechnung}
\label{sec:Fehlerrechnung}
Für die Fehlerrechnung werden folgende Formeln aus der Vorlesung verwendet.
für den Mittelwert gilt
\begin{equation}
    \overline{x}=\frac{1}{N}\sum_{i=1}^N x_i ß\; \;\text{mit der Anzahl N und den Messwerten x} 
    \label{eqn:Mittelwert}
\end{equation}
Der Fehler für den Mittelwert lässt sich gemäß
\begin{equation}
    \increment \overline{x}=\frac{1}{\sqrt{N}}\sqrt{\frac{1}{N-1}\sum_{i=1}^N(x_i-\overline{x})^2}
    \label{eqn:FehlerMittelwert}
\end{equation}
berechnen.
Wenn im weiteren Verlauf der Berechnung mit der fehlerhaften Größe gerechnet wird, kann der Fehler der folgenden Größe
mittels Gaußscher Fehlerfortpflanzung berechnet werden. Die Formel hierfür ist
\begin{equation}
    \increment f= \sqrt{\sum_{i=1}^N\left(\frac{\partial f}{\partial x_i}\right)^2\cdot(\increment x_i)^2}.
    \label{eqn:GaussMittelwert}
\end{equation}
\subsection{Auswertung der Messergebnisse}

Die Auswertung kann in zwei Teile unterteilt werden. Im ersten Teil wird die Wellenlänge $\lambda$ des Lasers bestimmt. Im zweiten Teil wird
der Brechungsindex von Luft untersucht.

\subsubsection{Berechnung der Wellenlänge des verwendeten Lasers.}
\label{sec:Wellenlänge}
Die 10 aufgenommenen Messwerte sind in \autoref{tab:Werte1} dargestellt. Aus jedem dieser Werte wird durch Umstellung von \autoref{eqn:Deltad} nach $\lambda$ ein Wert für die Wellenlänge bestimmt. 
Dabei gilt es zu beachten, dass bei der Berechnung der Hebelarm mit berücksichtigt werden muss. Um tatsächliche Verschiebung zu erhalten muss somit die gemessene Auslenkung mit dem Faktor $\frac{1}{5.046}$ multipliziert werden.% \textbf{oder Dividiert?????????}.
Aufgrund der Tatsache, dass während Messung 3 das Messgerät kurzzeitig ausgefallen ist, ist dieser Wert wesentlich geringer und wird deshalb weder in die Berechnung der Standardabweichung noch die des Durchschnitts der Wellenlänge einfließen.
Die Messgenauigkeit der Messschraube liegt bei $\pm 1.5\, \unit{\micro \meter}$\cite{Messgenauigkeit}. Dieser Fehler ist jedoch zum Ablesefehler, welcher sich auf $\pm 0.01\, \unit{\milli \meter}$ beläuft, nur etwa $15 \%$ so groß und deshalb vernachlässigbar.
Der Fehler der detektierten Maxima wird über die Standardabweichung errechnet. Hierzu wird \autoref{eqn:Standardabweichung} verwendet. Die Abweichung der Wellenlänge wird über die Gaußsche Fehlerfortpflanzung aus \autoref{eqn:GaussMittelwert} berechnet. 
\begin{table}
    \centering
    \caption{Messwerte zur Bestimmung der Wellenlänge $\lambda$}
    \begin{tabular}{c c | c}
        \toprule
        $d \mathrm{/} 10^{-3}\, \unit{\meter}$ & $z$: Anzahl der detektierten Maxima & $\lambda \mathrm{/} 10^{-9}\, \unit{\meter}$\\
        \midrule
        5.00 \pm 0.01 & 2540.00\pm 181.98& 780.22\pm 1.57\\
        5.00 \pm 0.01 & 2607.00\pm 181.98 & 760.17\pm 1.52\\
        5.00 \pm 0.01 & 2117.00\pm 181.98 & 936.12\pm 1.87\\
        5.00 \pm 0.01 & 3002.00\pm 181.98 & 660.15\pm 1.32\\
        5.00 \pm 0.01 & 2504.00\pm 181.98 & 791.44\pm 1.58\\
        5.00 \pm 0.01 & 2519.00\pm 181.98 & 786.73\pm 1.57\\
        5.00 \pm 0.01 & 2836.00\pm 181.98 & 698.79\pm 1.40\\
        5.00 \pm 0.01 & 2596.00\pm 181.98 & 763.39\pm 1.53\\
        5.00 \pm 0.01 & 2664.00\pm 181.98 & 743.91\pm 1.49\\
        5.00 \pm 0.01 & 2967.00\pm 181.98 & 667.93\pm 1.34\\
        \bottomrule
    \end{tabular}
    \label{tab:Werte1}
\end{table}
Aus den berechneten Einzelwerten kann so über \autoref{eqn:Mittelwert} und \autoref{eqn:FehlerMittelwert} ein Mittelwert und dessen Fehler beestimmt werden.
So ergibt sich für 
\begin{equation*}
    \bar{\lambda}_{\text{exp}}= (736.0\pm 1.5)\, 10^{-9}\, \unit{\meter}\, .
\end{equation*}


\newpage
\subsubsection{Bestimmung des Brechungsindex von Luft}
\label{sec:nvonLuft}
Der Brechungsindex wird mittels \autoref{eqn:brechung} berechnet und gemeinsam mit der Zählrate $z$ in \autoref{tab:brechung} dargestellt. Dabei wurden 
für die Normalbedingungen, die Temperatur und die Größe der Messzelle
\begin{align*}
    \text{Normaldruck: }p_0 &= 1.0132 \,\unit{\bar}\; ,\\
    \text{Normaltemperatur: }T_0 &= 273.15 \,\unit{\kelvin}\; ,\\
    \text{Umgebungstemperatur: }T &= 293.15 \,\unit{\kelvin}\; , \\
    \text{Größe der Messzelle: }b &= 50\,\unit{\milli\meter} \\
\end{align*}
angenommen. Auch hier wurde die Abweichung der Messwerte mittels der Standardabweichung (\autoref{eqn:Standardabweichung}) berechnet.
\begin{table}
    \centering
    \caption{Messwerte zur Bestimmung des Brechungsindex $n$.}
    \begin{tabular}{c c}
        \toprule
        $z$: Anzahl der detektierten Maxima & $n$\\
        \midrule
        30 \pm 4.42 & $1.00058 \pm 1.18 \cdot 10^{-6}$\\
        17 \pm 4.42 & $1.00033\pm 6.71 \cdot 10^{-7}$\\
        19 \pm 4.42 & $1.00037\pm 7.50 \cdot 10^{-7} $\\
        21 \pm 4.42 & $1.00041\pm 8.29 \cdot 10^{-7}$ \\
        24 \pm 4.42 & $1.00046 \pm 9.48 \cdot 10^{-7}$ \\
       
        \bottomrule
    \end{tabular}
    \label{tab:brechung}
\end{table}

Der Mittelwert des Brechungsindex ergibt sich zu
\begin{equation*}
    \bar{n} = 1.0004301 \pm 0.0000009 \; .
\end{equation*}