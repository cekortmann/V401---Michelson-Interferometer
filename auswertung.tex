\section{Auswertung}
\label{sec:Auswertung}

\subsection{Fehlerrechnung}
\label{sec:Fehlerrechnung}
Für die Fehlerrechnung werden folgende Formeln aus der Vorlesung verwendet.
für den Mittelwert gilt
\begin{equation}
    \overline{x}=\frac{1}{N}\sum_{i=1}^N x_i ß\; \;\text{mit der Anzahl N und den Messwerten x} 
    \label{eqn:Mittelwert}
\end{equation}
Der Fehler für den Mittelwert lässt sich gemäß
\begin{equation}
    \increment \overline{x}=\frac{1}{\sqrt{N}}\sqrt{\frac{1}{N-1}\sum_{i=1}^N(x_i-\overline{x})^2}
    \label{eqn:FehlerMittelwert}
\end{equation}
berechnen.
Wenn im weiteren Verlauf der Berechnung mit der fehlerhaften Größe gerechnet wird, kann der Fehler der folgenden Größe
mittels Gaußscher Fehlerfortpflanzung berechnet werden. Die Formel hierfür ist
\begin{equation}
    \increment f= \sqrt{\sum_{i=1}^N\left(\frac{\partial f}{\partial x_i}\right)^2\cdot(\increment x_i)^2}.
    \label{eqn:GaussMittelwert}
\end{equation}

Die Auswertung kann in zwei Teile unterteilt werden. Im ersten Teil wird die Wellenlänge $\lambda$ des Lasers bestimmt. Im zweiten Teil wird
der Brechungsindex von Luft untersucht.

\subsection{Berechnung der Wellenlänge des verwendeten Lasers.}
\label{sec:Wellenlänge}
Die 10 aufgenommenen Messwerte sind in \autoref{tab:Werte1} dargestellt. Aus jedem dieser Werte wird durch Umstellung von \autoref{eqn:Deltad} nach $\lambda$ ein Wert für die Wellenlänge bestimmt. 
Dabei gilt es zu beachten, dass bei der Berechnung der Hebelarm mit berücksichtigt werden muss. Um tatsächliche Verschiebung zu erhalten muss somit die gemessene Auslenkung mit dem Faktor $5.046$ multipliziert werden \textbf{oder Dividiert?????????}.
%Aufgrund der Tatsache, dass während Messung 3 das Messgerät kurzzeitig ausgefallen ist, ist dieser Wert wesentlich geringer und wird aufgrund dessen weder in die Berechnung der Standardabweichung noch die des Durchschnitts der Wellenlänge einfließen.
Die Messgenauigkeit der Messschraube liegt bei $\pm 1.5\, \unit{\micro \meter}$\cite{Messgenauigkeit}. Dieser Fehler ist jedoch zum Ablesefehler, welcher sich auf $\pm 0.01\, \unit{\milli \meter}$ beläuft, nur etwa $15 \%$ so groß und deshalb vernachlässigbar.
\begin{table}
    \centering
    \caption{Messwerte zur Bestimmung der Wellenlänge $\lambda$}
    \begin{tabular}{c c | c}
        \toprule
        $d \mathrm{/} 10^{-3}\, \unit{\meter}$ & $z$: Anzahl der detektierten Maxima & $\lambda \mathrm{/} 10^{-9}\, \unit{\meter}$\\
        \midrule
        5.00 \pm 0.01 & 2540.00\pm 181.98& \\
        5.00 \pm 0.01 & 2607.00\pm 181.98 & \\
        5.00 \pm 0.01 & 2117.00\pm 181.98 & \\
        5.00 \pm 0.01 & 3002.00\pm 181.98 & \\
        5.00 \pm 0.01 & 2504.00\pm 181.98 & \\
        5.00 \pm 0.01 & 2519.00\pm 181.98 & \\
        5.00 \pm 0.01 & 2836.00\pm 181.98 & \\
        5.00 \pm 0.01 & 2596.00\pm 181.98 & \\
        5.00 \pm 0.01 & 2664.00\pm 181.98 & \\
        5.00 \pm 0.01 & 2967.00\pm 181.98 & \\
        \bottomrule
    \end{tabular}
    \label{tab:Werte1}
\end{table}