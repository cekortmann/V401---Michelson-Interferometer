\section{Durchführung}
\label{sec:Durchführung}

Nach der Justierung, welche dazu dient ein möglichst deutliches Interferenzmuster zu erzeugen, werden zwei unterschiedliche Messungen durchgeführt.

Die erste besteht darin, die Wellenlänge des verwendeten LAsers auszurechnen. Hierzu werden Laser und Detektor eingeschaltet. Im Anschluss wird ein Start- und Endwert auf der Mikrometerscharaube festgelegt.
Der Motor wird eingeschaltet und die Messung beginnt. Der Motor verändert so den Winkel des Spiegels und wird beim Erreichen des Endwertes gestoppt. Die Anzahl der detektierten Impulse wird notiert, der MOtor umgepolt sodass die zweite Messung die Winkelneigung rückwärts abläuft.
So werden insgesamt fünf Doppelmessungen vorgenommen.

Durch die zweite Messung soll der Brechungsindex der Luft bestimmt werden. Hierzu wir die Gaskammer evakuiert. Während des Abpumpens der Luft, läuft der Detektor und notiert die Anzahl der Impulse.
Ist die Kammer so luftleer, wie es die Konstruktion zulässt, wird die Anzahl notiert. Im Anschluss wird die Kammer mit Luft geflutet, wenn die Kammer erneut atmosphärischen Druck erreicht hat, wird auch diese Anzahl der
detektierten Impulse notiert. Diese Messung insgesamt fünf mal durchgeführt.
